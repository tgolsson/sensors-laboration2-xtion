\usepackage{graphicx}
\usepackage{caption}
\usepackage[a4paper, top=1in, bottom=1.1in, left=1in, right=1in]{geometry}
\usepackage[utf8]{inputenc} % utf8
\usepackage[T1]{fontenc}
\usepackage{xcolor}
\usepackage{listings}
\usepackage{subcaption}
\usepackage{siunitx}\usepackage{wrapfig}
\usepackage{varioref}\usepackage{amsmath}
\usepackage{commath}
\usepackage{mathtools}
\usepackage{hyperref}
\usepackage{url}
\usepackage{attachfile}
\usepackage{booktabs}
\usepackage[toc,page]{appendix}
\setlength{\belowcaptionskip}{-6pt}
\makeatletter
\lst@Key{matchrangestart}{f}{\lstKV@SetIf{#1}\lst@ifmatchrangestart}
\def\lst@SkipToFirst{%
  \lst@ifmatchrangestart\c@lstnumber=\numexpr-1+\lst@firstline\fi
  \ifnum \lst@lineno<\lst@firstline
  \def\lst@next{\lst@BeginDropInput\lst@Pmode
    \lst@Let{13}\lst@MSkipToFirst
    \lst@Let{10}\lst@MSkipToFirst}%
  \expandafter\lst@next
  \else
  \expandafter\lst@BOLGobble
  \fi}
\makeatother

\lstset{  
  backgroundcolor=\color{gray!30},   % choose the background color; you must add \usepackage{color} or \usepackage{xcolor}
  basicstyle=\scriptsize,        % the size of the fonts that are used for the code
  breakatwhitespace=false,         % sets if automatic breaks should only happen at whitespace
  breaklines=true,                 % sets automatic line breaking
  captionpos=t,                    % sets the caption-position to bottom
  escapeinside={\%*}{*)},          % if you want to add LaTeX within your code
  extendedchars=true,              % lets you use non-ASCII characters; for 8-bits encodings only, does not work with UTF-8
  frame=single,                   % adds a frame around the code
  keepspaces=true,                 % keeps spaces in text, useful for keeping indentation of code (possibly needs columns=flexible)
  keywordstyle=\color{blue},       % keyword style
  language=C++,                 % the language of the code
  numbers=left,                    % where to put the line-numbers; possible values are (none, left, right)
  numbersep=20pt,                   % how far the line-numbers are from the code
  numberstyle=\tiny\color{gray}, % the style that is used for the line-numbers
  rulecolor=\color{blue!20},       
  showspaces=false,                % show spaces everywhere adding particular underscores; it overrides 'showstringspaces'
  showstringspaces=false,          % underline spaces within strings only
  showtabs=false,                  % show tabs within strings adding particular underscores
  stepnumber=1,                    % the step between two line-numbers. If it's 1, each line will be numbered
  tabsize=2,                   % sets default tabsize to 2 spaces
  framesep=7pt,
  xleftmargin=12pt,
  xrightmargin=11pt
}


\setlength{\fboxsep}{4pt}
\DeclareCaptionFormat{myformat}{%
  \hspace{1pt}\fcolorbox{blue!20}{gray!20}{\footnotesize\parbox{\dimexpr\textwidth-17pt\relax}{#1#2\ttfamily#3}}\vspace{-4pt}
}
\captionsetup[lstlisting]{format=myformat}

\captionsetup[figure]{labelfont=sf,hypcap=false,format=hang,margin=0.5cm,justification=RaggedRight,calcwidth=0.7\linewidth,font=footnotesize,justification=justified}
\captionsetup[subfigure]{labelfont=sf,hypcap=false,format=hang,margin=0.5cm,justification=RaggedRight,calcwidth=0.7\linewidth,font=footnotesize,justification=justified}
\captionsetup[table]{labelfont=sf,hypcap=false,format=hang,margin=1cm,justification=RaggedRight,calcwidth=0.8\linewidth,font=footnotesize,justification=justified}
\labelformat{equation}{(#1)}

%%% Math typesetting macros
\newcommand{\di}[2]{#1_\textup{#2}} % Descriptive Index: Macro for quick upright index (as opposed to a variable index, which should be italic)


\renewcommand{\lstlistlistingname}{Code listings}
\bibliographystyle{ieeetr}


%%%%%%%%%%%%%%%%%%%%%%%%%%%%%%
%%% STOLEN FROM STACKOVERFLOW
%%%%%%%%%%%%%%%%%%%%%%%%%%%%%%

\newcommand\YAMLcolonstyle{\color{red}\mdseries}
\newcommand\YAMLkeystyle{\color{black}\bfseries}
\newcommand\YAMLvaluestyle{\color{blue}\mdseries}

\makeatletter

% here is a macro expanding to the name of the language
% (handy if you decide to change it further down the road)
\newcommand\language@yaml{yaml}

\expandafter\expandafter\expandafter\lstdefinelanguage
\expandafter{\language@yaml}
{
  keywords={true,false,null,y,n},
  keywordstyle=\color{darkgray}\bfseries,
  basicstyle=\YAMLkeystyle,                                 % assuming a key comes first
  sensitive=false,
  comment=[l]{\#},
  morecomment=[s]{/*}{*/},
  commentstyle=\color{purple}\ttfamily,
  stringstyle=\YAMLvaluestyle\ttfamily,
  moredelim=[l][\color{orange}]{\&},
  moredelim=[l][\color{magenta}]{*},
  moredelim=**[il][\YAMLcolonstyle{:}\YAMLvaluestyle]{:},   % switch to value style at :
  morestring=[b]',
  morestring=[b]'',
  literate =    {---}{{\ProcessThreeDashes}}3
  {>}{{\textcolor{red}\textgreater}}1
  {|}{{\textcolor{red}\textbar}}1
  {\ -\ }{{\mdseries\ -\ }}3,
}

% switch to key style at EOL
\lst@AddToHook{EveryLine}{\ifx\lst@language\language@yaml\YAMLkeystyle\fi}
\makeatother

\newcommand\ProcessThreeDashes{\llap{\color{cyan}\mdseries-{-}-}}

%%% Local Variables:
%%% mode: latex
%%% TeX-master: t
%%% End:
