\documentclass[11pt]{article}

%% Separate file for preamble with macros and stuff
\include{preamble} 

%% For make-title
\title{Laboration 2: RGBD-cameras\\ {\small Sensors and Sensing}}
\author{Marek Bečica, Tom Olsson}
\date{\today}

\begin{document}
\maketitle %Title area
\begin{center}
  \emph{All code for this exercise can be found at \\ \url{https://github.com/tgolsson/sensors-laboration2-xtion}}
\end{center}
\lstlistoflistings % List of all code snippets
\listoffigures % List of all figures
\listoftables
\lstset{ matchrangestart=t} %initialise the linerange-macro for \lstinput...
\section{Theory and motivation}
\subsection{RGBD-cameras}
% TODO: RGBD, 

RGBD-cameras, short for \emph{Red-Green-Blue-Depth}-camera, is a type of low-cost camera commonly used for robot vision. The concept became widely popular with the release of the Microsoft Kinect in late 2010. \par

These cameras consist of two separate parts: one normal color-based camera, and one infra-red sensor with accompanying projector. The sensing consists of projecting a deterministic pattern onto the scene, and then unprojecting them by comparing to previously captured patterns at known depths. By interpolating through these patterns, a full depth-image is generated.  
\subsection{Noise}
% TODO: Noise and smoothing; object identification
A common problem in any type of sensing is the introduction of noise into the system. This noise can come from many sources, and be predictable or unpredictable. Examples of noise sources could be frequency hum from electric circuits, flickering lights, air pollution or pure inaccuracy. This noise can skew the results of sensors that make algorithm much more error prone. \par

There are many approaches to reduce noise. Proper calibration and good testing environments is a good start, but this can only reduce external noise. Internal noise of the sensor needs to be analyzed and minimized on a much lower-level such as by using specially constructed algorithms. For sensors, that generate some sort of sequence one very naive (but nonetheless effective) approach is the use of smoothing. \par

\section{Implementation}
The purpose of this exercise is to calibrate an RGBD-camera and investigate its characteristics. Then, several smoothing algorithms shall be evaluated for the depth data.
\subsection{Hardware and environment}
This exercise was performed using an \emph{ASUS Xtion Pro}. The camera was connected over \emph{USB2} to a laptop running Linux kernel 4.2.5. The communication to the camera is done using the \emph{Robot Operating System} [ROS] version \emph{Indigo Igloo}. All packages used are compiled directly from GitHub development branch for Indigo Igloo. \par
Other software used includes the OpenCV libraries, version 2.4.12.2-1.
% TODO: Tom
\subsection{Task 1}
% TODO: Marek : Screenshots, short introduction
\subsection{Task 2}
% TODO: Tom : Sample files, point cloud & images
\subsection{Task 3}
% TODO: Marek : Calibration file & process
\subsection{Task 4}
% TODO: Tom : Show plots; Measuring setup;
% TODO: Both : Analysis of plots
\subsection{Task 5}
% TODO: Tom : Implementation, challenges (NAAAN), push code with visible images
% TODO: Marek : Target scene (point 3)
% TODO: Both : Analysis

\section{Results}
% Both : Summary...

\bibliography{References}
\end{document}



%%% Local Variables:
%%% mode: latex
%%% TeX-master: t
%%% End:
